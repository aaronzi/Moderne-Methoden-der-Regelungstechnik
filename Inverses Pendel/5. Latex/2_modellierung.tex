\section{Modellierung: Energiemethode nach Lagrange}

\subsection{Ansatz}

Die nachfolgende Gleichung zeigt den Lagrange Ansatz unter Berücksichtigung der dissipativen Funktion. Diese besagt in Erweiterung zu der Lagrange-Formulierung, dass Energie in einem Vorgang in Wärme umgewandelt wird. Mit Hilfe der dissipativen Funktion können Reibungsverluste bei der Energiemethode nach Lagrange berücksichtigt werden.

\begin{align} \label{eq:Gleichung1}
    \frac{d}{dt} \left(\frac{\partial L}{\partial \dot{q_{\mathrm{i}}}}\right) - \frac{\partial L}{\partial q_{\mathrm{i}}} + \frac{\partial D}{\partial \dot{q_{\mathrm{i}}}} = F_{\mathrm{i}}
\end{align}

\subsection{Freiheitsgrade und Zwangsbedingungen}

In \autoref{fig:Bild1} sind zwei Massepunkte im $\mathbb{R}^2$ zu erkennen. Somit gilt grundsätzlich:
\begin{itemize}
    \item 2 Punkte: 4 Freiheitsgrade (FHG)
\end{itemize}

Das inverse Pendel besitzt jedoch auch zwei Zwangsbedingungen, die wie folgt formuliert werden können:

\begin{itemize}
    \item Der Wagen kann sich nur horizontal bewegen: \\ $y_{\mathrm{M}} = 0$
    \item Die Masse $m$ am Ende des Pendels ist mit dem Wagen gekoppelt: \\ $(y_{\mathrm{M}} - y_{\mathrm{m}})^2 + (x_{\mathrm{M}} - x_{\mathrm{m}})^2 = l^2$
\end{itemize}

Somit bleiben am Ende noch zwei Freiheitsgrade (FHG) übrig.

\subsection{Generalisierte Koordinaten}

Aus den verbliebenen Freiheitsgraden werden die beiden generalisierten Koordinaten abgeleitet.

\begin{itemize}
    \item $q_{\mathrm{1}} = x_{\mathrm{M}}$
    \item $q_{\mathrm{2}} = \varphi$
\end{itemize}

\subsection{Berechnung der kinetischen und potentiellen Energie}

Der Ansatz zur Berechnung einer kinetischen Energie ist nachfolgend gezeigt.

\begin{align}\label{eq:Gleichung2}
    E_{\mathrm{kin}} = \frac{1}{2} \cdot m \cdot v^2
\end{align}

Zu berücksichtigen ist, dass beide Massen ($m$ und $M$) eine kinetische Energie besitzen (\autoref{eq:Gleichung3} und \autoref{eq:Gleichung4}).

\begin{align}
    E_{\mathrm{kin}} &= \frac{1}{2} \cdot M \cdot \dot{x}_{\mathrm{M}}^2 + \frac{1}{2} \cdot m \cdot v_{\mathrm{m}}^2 \label{eq:Gleichung3} \\
    E_{\mathrm{kin}} &= \frac{1}{2} \cdot M \cdot \dot{x}_{\mathrm{M}}^2 + \frac{1}{2} \cdot m \cdot \left(\dot{x}_{\mathrm{m}}^2 + \dot{y}_{\mathrm{m}}^2\right) \label{eq:Gleichung4}
\end{align}

Weiter gilt:

\begin{align*}
    x_{\mathrm{m}} &= x_{\mathrm{M}} + l \cdot \sin({\varphi}) \\
    y_{\mathrm{m}} &= l \cdot \cos({\varphi}) \\
    \dot{x_{\mathrm{m}}} &= \dot{x_{\mathrm{M}}} + l \cdot \dot{\varphi} \cdot \cos{\varphi} \\
    \dot{y_{\mathrm{m}}} &= -l \cdot \dot{\varphi} \cdot \sin({\varphi})
\end{align*}

Daraus resultiert:

\begin{dmath*}
    E_{\mathrm{kin}} = \frac{1}{2} \cdot M \cdot \dot{x}_{\mathrm{M}}^2 + \frac{1}{2} \cdot m \cdot \left(\left(\dot{x}_{\mathrm{M}} + l \cdot \dot{\varphi} \cdot \cos({\varphi})\right)^2 + \left( -l \cdot \dot{\varphi} \cdot \sin({\varphi})\right)^2\right) \\
    = \frac{1}{2} \cdot M \cdot \dot{x}_{\mathrm{M}}^2 + \frac{1}{2} \cdot m \cdot \left(\dot{x}_{\mathrm{M}}^2 + 2 \cdot \dot{x}_{\mathrm{M}} \cdot l \cdot \dot{\varphi} \cdot \cos({\varphi}) + l^2 \cdot \dot{\varphi}^2 \cdot \cos^2(\varphi) + l^2 \cdot \dot{\varphi}^2 \cdot \sin^2(\varphi)\right) \\
    = \frac{1}{2} \cdot M \cdot \dot{x}_{\mathrm{M}}^2 + \frac{1}{2} \cdot m \cdot \dot{x}_{\mathrm{M}}^2 + m \cdot \dot{x}_{\mathrm{M}} \cdot l \cdot \dot{\varphi} \cdot \cos({\varphi}) + \frac{1}{2} \cdot m \cdot l^2 \cdot \dot{\varphi}^2 \cdot \cos^2(\varphi) + \frac{1}{2} \cdot m \cdot l^2 \cdot \dot{\varphi}^2 \cdot \sin^2(\varphi)
\end{dmath*}

Durch das Zusammenfassen der vorangegangenen Beziehung folgt \autoref{eq:Gleichung5} für die gesamte kinetische Energie des Systems.

\begin{align} \label{eq:Gleichung5}
    E_{\mathrm{kin}} = \frac{1}{2} \cdot \dot{x}_{\mathrm{M}}^2 \cdot (M + m) + \frac{1}{2} \cdot m \cdot \left( 2 \cdot \dot{x}_{\mathrm{M}} \cdot l \cdot \dot{\varphi} \cdot \cos({\varphi}) + l^2 \cdot \dot{\varphi}^2\right)
\end{align}

Ausschließlich die Masse $m$ am Pendelende besitzt eine für den Lagrange-Formalismus relevante potentielle Energie (\autoref{eq:Gleichung6}).

\begin{align}
    E_{\mathrm{pot}} &= m \cdot g \cdot h \nonumber \\
    E_{\mathrm{pot}} &= m \cdot g \cdot y_{\mathrm{m}} \nonumber \\
    E_{\mathrm{pot}} &= m \cdot g \cdot l \cdot \cos({\varphi}) \label{eq:Gleichung6}
\end{align}

\subsection{Herleitung der Bewegungsgleichungen}

Die Lagrange-Funktion wird aus der Differenz der kinetischen und der potentiellen Energie berechnet (\autoref{eq:Gleichung7}).

\begin{align} 
        L &= E_{\mathrm{kin}} - E_{\mathrm{pot}}  \label{eq:Gleichung7} \\ 
        L &= \frac{1}{2} \cdot \dot{x}_{\mathrm{M}}^2 \cdot (M + m) + \frac{1}{2} \cdot m \cdot \left( 2 \cdot \dot{x}_{\mathrm{M}} \cdot l \cdot \dot{\varphi} \cdot \cos({\varphi}) + l^2 \cdot \dot{\varphi}^2\right) - m \cdot g \cdot l \cdot \cos({\varphi}) \label{eq:Gleichung8}
\end{align}

Im ersten Schritt wird die Bewegungsgleichung des Wagens hergeleitet. Dafür wird zunächst \autoref{eq:Gleichung8} nach der ersten zeitlichen Ableitung der generalisierten Koordinate $x_{\mathrm{M}}$ partiell differenziert:

\begin{align}\label{eq:Gleichung9}
    \frac{\partial L}{\partial \dot{x}_{\mathrm{M}}} = (M + m) \cdot \dot{x}_{\mathrm{M}} + m \cdot l \cdot \dot{\varphi} \cdot \cos(\varphi)
\end{align}

Die vorangegangene Gleichung wird zeitlich differenziert:

\begin{align}\label{eq:Gleichung10}
    \frac{d}{dt}\left(\frac{\partial L}{\partial \dot{x}_{\mathrm{M}}}\right) = (M + m) \cdot \ddot{x}_{\mathrm{M}} + m \cdot l \cdot \left(\ddot{\varphi} \cdot \cos(\varphi) - \dot{\varphi}^2 \cdot \sin(\varphi) \right)
\end{align}

Im zweiten Schritt wird die Lagrange-Funktion nach der generalisierten Koordinate $x_{\mathrm{M}}$ abgeleitet.

\begin{align}\label{eq:Gleichung11}
    \frac{\partial L}{\partial x_{\mathrm{M}}} = 0
\end{align}

Abschließend wird die dissipative Funktion nach der 1. Ableitung der generalisierten Koordinate $x_{\mathrm{M}}$ differenziert.

\begin{align}\label{eq:Gleichung12}
    \frac{\partial D}{\partial \dot{x_{\mathrm{M}}}} = d \cdot \dot{x_{\mathrm{M}}}
\end{align}

Durch das Einsetzen der \autoref{eq:Gleichung9} bis \autoref{eq:Gleichung12} in den Ansatz aus \autoref{eq:Gleichung1} resultiert die vollständige Bewegungsgleichung des Wagens.

\begin{align}\label{eq:Gleichung13}
    (M + m) \cdot \ddot x_{\mathrm{M}} + m \cdot l \cdot \left( \ddot \varphi \cdot \cos({\varphi}) - \dot \varphi^2 \cdot \sin({\varphi})\right) + d \cdot \dot x_{\mathrm{M}} = F_{\mathrm{a}}
\end{align}

Analog wird die Bewegungsgleichung des Pendels entwickelt. Hierbei wird zunächst \autoref{eq:Gleichung8} nach der 1. Ableitung der generalisierten Koordinate $\varphi$ partiell differenziert.

\begin{align}\label{eq:Gleichung14}
    \frac{\partial L}{\partial \dot{\varphi}} = m \cdot l \cdot \dot{x}_{\mathrm{M}} \cdot \cos(\varphi) + m \cdot l^2 \cdot \dot{\varphi}
\end{align}

Die vorangegangene Gleichung wird nach der Zeit abgeleitet:

\begin{align}\label{eq:Gleichung15}
    \frac{d}{dt}\left(\frac{\partial L}{\partial \dot{\varphi}}\right) = m \cdot l \cdot \left(\ddot{x}_{\mathrm{M}} \cdot  \cos(\varphi) - \dot{x}_{\mathrm{M}} \cdot \sin(\varphi) \cdot \dot{\varphi}\right) + m \cdot l^2 \cdot \ddot{\varphi}
\end{align}

Als nächstes wird für die Bewegungsgleichung des Pendels die Lagrange-Funktion nach der generalisierten Koordinate $\varphi$ abgeleitet:

\begin{align}\label{eq:Gleichung16}
    \frac{\partial L}{\partial \varphi} = -m \cdot l \cdot \dot{\varphi} \cdot \dot{x}_{\mathrm{M}} \cdot \sin(\varphi) + m \cdot g \cdot l \cdot \sin(\varphi)
\end{align}

Abschließend wird die dissipative Funktion analog nach der 1. Ableitung der generalisierten Koordinate $\varphi$ differenziert:

\begin{align}\label{eq:Gleichung17}
    \frac{\partial D}{\partial \dot{\varphi}} = d_{\mathrm{Mf}} \cdot \dot{\varphi}
\end{align}

Über das Anwenden von \autoref{eq:Gleichung1} folgt die Bewegungsgleichung des Pendels zu:

\begin{align}\label{eq:Gleichung18}
    \ddot{x}_{\mathrm{M}} \cdot \cos(\varphi) + l \cdot \ddot{\varphi} - g \cdot \sin(\varphi) + \frac{d_{\mathrm{Mf}} \cdot \dot{\varphi}}{m \cdot l} = 0
\end{align}