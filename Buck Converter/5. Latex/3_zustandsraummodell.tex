\section{Zustandsraumdarstellung} \label{sec:Zustandsraumdarstellung}
Um das Verhalten mittels mathematischer Beziehungen zu veranschaulichen, wird die Zustandsraumdarstellung verwendet. Folgende messbaren Ein- und Ausgänge werden festgelegt:\\
\newline
Systemeingänge:
\begin{align*}
    \underline{u} &= d
\end{align*}
\newline
Systemzustände:
\begin{align}
    \underline{x} &=
    \begin{bmatrix}
        x_{\mathrm{1}} \\
        x_{\mathrm{2}}
    \end{bmatrix} = 
    \begin{bmatrix}
        v_{\mathrm{PV}} \\
        i_{\mathrm{L}}
    \end{bmatrix}
    \label{eq:Gleichung9}
\end{align}
\newline
Die Ausgänge des Systems gleichen den beiden Zuständen.\\
\newline
Systemausgänge:
\begin{align*}
    \underline{y} &= 
    \begin{bmatrix}
        v_{\mathrm{PV}} \\
        i_{\mathrm{L}}
    \end{bmatrix}
\end{align*}

\subsection{Nichtlinear}
\label{sec:Nichtlinear}
Die nachstehend gezeigten Zustandsänderungsgleichungen des Mittelwertmodells werden für weitere Betrachtungen als bekannt vorausgesetzt und deren Herleitung als korrekt angenommen.\\
\newline
Mittelwertmodell:
\begin{align*}
    \dot{x}_{\mathrm{1}} &= \frac{1}{C}\cdot i_{\mathrm{pv}}(x_{\mathrm{1}}, S, T_{\mathrm{c}})-\frac{1}{C}\cdot x_{\mathrm{2}}\cdot d \\\\
    \dot{x}_{\mathrm{2}} &= \frac{1}{L}\cdot x_{\mathrm{1}}\cdot d-\frac{1}{L}\cdot v_{\mathrm{DC}}
\end{align*}
\newline
mit:
\begin{align*}
    L = 1.76mH;\quad C = 15.7mF;\quad d = 0.8579;\quad v_{\mathrm{DC}} = 900V;\quad S = 1000\frac{W}{m^2};\quad T_{\mathrm{c}} = 298K
\end{align*}

\subsection{Linear}
\label{sec:Linear}
\subsubsection{Ruhelagen}
\label{sec:Ruhelagen}
Die Ruhelagen des Systems werden ermittlen, indem das Mittelwertmodell aus \autoref{sec:Nichtlinear} gleich Null gesetzt wird und anschließend die Auflösung nach den Zuständen $x_{\mathrm{1}}$ und $x_{\mathrm{2}}$ erfolgt.\\
\newline
Vorgabe:
\begin{align*}
    \dot{x}_{\mathrm{1}}^* &= \dot{x}_{\mathrm{1}}(x_{\mathrm{1}}^*, x_{\mathrm{2}}^*) \overset{!}{=} 0 \\\\
    \dot{x}_{\mathrm{2}}^* &= \dot{x}_{\mathrm{2}}(x_{\mathrm{1}}^*, x_{\mathrm{2}}^*) \overset{!}{=} 0
\end{align*}
\newline
Bestimmung von $\dot{x}_{\mathrm{1}}^*$:
\begin{align}
    0 &= \frac{1}{L}\cdot x_{\mathrm{1}}^*\cdot d-\frac{1}{L}\cdot v_{\mathrm{DC}} \nonumber \\ \nonumber \\
    \frac{1}{L}\cdot v_{\mathrm{DC}} &= \frac{1}{L}\cdot x_{\mathrm{1}}^*\cdot d \nonumber \\ \nonumber \\
    x_{\mathrm{1}}^* &= \frac{v_{\mathrm{DC}}}{d}
    \label{eq:Gleichung10}
\end{align}
\newline
Bestimmung von $\dot{x}_{\mathrm{2}}^*$:
\begin{align}
    0 &= \frac{1}{C}\cdot i_{\mathrm{pv}}(x_{\mathrm{1}}^*, S, T_{\mathrm{c}})-\frac{1}{C}\cdot x_{\mathrm{2}}^*\cdot d \nonumber \\ \nonumber \\
    \frac{1}{C}\cdot x_{\mathrm{2}}^*\cdot d &= \frac{1}{C}\cdot i_{\mathrm{pv}}(x_{\mathrm{1}}^*, S, T_{\mathrm{c}}) \nonumber \\ \nonumber \\
    \dot{x}_{\mathrm{2}}^* &= \frac{i_{\mathrm{pv}}(x_{\mathrm{1}}^*, S, T_{\mathrm{c}})}{d}
    \label{eq:Gleichung11}
\end{align}
\clearpage
Die spezifischen Ruhelagen aufgrund der vorgegebenen Parameter resultieren zu:\\
\newline
Ruhelagen des Systems:
\begin{align}
    \underline{\underline{x_{\mathrm{1}}^*}} &= \frac{900V}{0.8579} \approx \underline{\underline{1049.13V}} \nonumber \\ \nonumber \\
    \underline{\underline{x_{\mathrm{2}}^*}} &= \frac{i_{\mathrm{pv}}(1049.13V, 1000\frac{W}{m^2}, 298K)}{0.8579} \approx\underline{\underline{3422.92A}} \nonumber \\ \nonumber \\
    \underline{x}^* &= 
    \begin{bmatrix}
        x_{\mathrm{1}}^* \\
        x_{\mathrm{2}}^*
    \end{bmatrix} \approx
    \begin{bmatrix}
        1049.13V \\
        3422.92A
    \end{bmatrix}
    \label{eq:Gleichung12}
\end{align}

\subsubsection{Linearisierungsvorschrift}
\label{sec:Linearisierungsvorschrift}
Die Vorschrift zur Linearisierung ist in \autoref{eq:Gleichung13} hinterlegt.\\
\newline
Linearisierungsvorschrift nach Taylor:
\begin{align}
    \begin{split}
        \dot{\underline{x}}^{*}+\Delta{\dot{\underline{x}}} &=\underline{f}(\underline{x}^{*}+\Delta{\underline{x}},\underline{u}^{*}+\Delta{\underline{u}})\\\\
        &=\underline{f}(\underline{x}^{*},\underline{u}^{*})+\left[\frac{\partial f_{\mathrm{i}}}{\partial x_{\mathrm{j}}}\right]_{(\underline{x}^{*}, \underline{u}^{*})}\cdot\Delta{\underline{x}}+\left[\frac{\partial f_{\mathrm{i}}}{\partial u_{\mathrm{j}}}\right]_{(\underline{x}^{*},\underline{u}^{*})}\cdot\Delta{\underline{u}}+\underline{R}(\Delta{\underline{x}^2}, \Delta{\underline{u}^2})
    \end{split}
    \label{eq:Gleichung13}
\end{align}
\newline
Durch die Annahme über das Verhalten der nichtlinearen Restglieder folgt die Struktur der Linearisierung aus \autoref{eq:Gleichung14}.\\
\newline
Zusammengefasste Struktur:
\begin{align}
    \begin{split}
        \Delta\dot{\underline{x}} &= \left[\frac{\partial f_{\mathrm{i}}}{\partial x_{\mathrm{j}}}\right]_{(\underline{x}^{*}, \underline{u}^{*})}\cdot\Delta{\underline{x}}+\left[\frac{\partial f_{\mathrm{i}}}{\partial u_{\mathrm{j}}}\right]_{(\underline{x}^{*},\underline{u}^{*})}\cdot\Delta{\underline{u}}\\\\
        \Delta{\underline{y}} &= \left[\frac{\partial h_{\mathrm{i}}}{\partial x_{\mathrm{j}}}\right]_{(\underline{x}^{*}, \underline{u}^{*})}\cdot\Delta{\underline{x}}+\left[\frac{\partial h_{\mathrm{i}}}{\partial u_{\mathrm{j}}}\right]_{(\underline{x}^{*},\underline{u}^{*})}\cdot\Delta{\underline{u}}
    \end{split}
    \label{eq:Gleichung14}
\end{align}

\subsubsection{Zustandsraummodell}
\label{sec:Zustandsraummodell}
Durch die Anwendung der Linearisierungsvorschrift aus \autoref{eq:Gleichung14} auf das Mittelwertmodell aus \autoref{sec:Nichtlinear} resultiert das linearisierte Zustandsraummodell zu:\\
Grundstruktur des linearisierten Zustandraummodells:
\begin{align}
    \begin{split}
    \Delta\underline{\dot{x}} &= 
    \begin{bmatrix}
        \frac{1}{C}\cdot\frac{\partial i_{\mathrm{pv}}(x_{\mathrm{1}}, S, T_{\mathrm{c}})}{\partial x_{\mathrm{1}}}\big |_{x_{\mathrm{1}}^*} & -\frac{1}{C}\cdot u^* \\\\
        \frac{1}{L}\cdot u^* & 0
    \end{bmatrix} \cdot \Delta \underline{x} +
    \begin{bmatrix}
        -\frac{1}{C}\cdot x_{\mathrm{2}}^* \\\\
        \frac{1}{L}\cdot x_{\mathrm{1}}^*
    \end{bmatrix} \cdot
    \Delta \underline{u}
     \\\\
    \Delta \underline{y} &= 
    \begin{bmatrix}
        1 & \quad 0 \\\\
        0 & \quad 1
    \end{bmatrix} \cdot \Delta \underline{x} + \underline{0} \cdot \Delta\underline{u}
    \end{split}
    \label{eq:Gleichung15}
\end{align}
\newline
mit:
\begin{align*}
    \Delta \underline{x} &= \underline{x} - \underline{x}_{\mathrm{c}} \\\\
    \Delta \underline{u} &= \Delta \underline{d} = \underline{d}_{\mathrm{dyn}}-\underline{d}
\end{align*}
\newline
Die Terme $\underline{x}_{\mathrm{c}}$ und $\underline{d}$ gleichen den Ruhelagen aus \autoref{eq:Gleichung12} und $d = 0.8579$. Das vollständige lineare Zustandsraummodell ist in \autoref{eq:Gleichung16} dargestellt und folgt durch die Berechnungen der einzelnen Terme aus \autoref{eq:Gleichung15}.\\
\newline
Linearisiertes Zustandsraummodell:
\begin{align}
    \begin{split}
    \Delta\underline{\dot{x}} &= 
    \begin{bmatrix}
        -150.4187 & -54.5419 \\\\
        486.5101 & 0
    \end{bmatrix} \cdot \Delta \underline{x} +
    \begin{bmatrix}
        -2.1763\cdot 10^5 \\\\
        5.9499\cdot 10^5
    \end{bmatrix} \cdot
    \Delta \underline{u} \\\\
    \Delta \underline{y} &= 
    \begin{bmatrix}
        1 & \quad 0 \\\\
        0 & \quad 1
    \end{bmatrix} \cdot \Delta \underline{x} + \underline{0} \cdot \Delta\underline{u}
    \end{split}
    \label{eq:Gleichung16}
\end{align}

\subsubsection{Überprüfung der Steuerbarkeit}
\label{sec:Ueberpruefung_der_Steuerbarkeit}
Zur Überprüfung der Steuerbarkeit wird im ersten Schritt die Steuerbarkeitsmatrix $Q_{\mathrm{s}}$ nach der Vorschrift aus \autoref{eq:Gleichung17} berechnet. Im Anschluss erfolgt die Berechnung der Determinante (quadratische Matrix) oder des Rangs (nichtquadratische Matrix).\\
\newline
Vorschrift:
\begin{align}
    \underline{Q}_{\mathrm{s}} &= \left(\underline{B} \quad \underline{A}\cdot\underline{B} \quad ... \quad \underline{A}^{(n-1)}\cdot\underline{B}\right)
    \label{eq:Gleichung17}
\end{align}
Steuerbarkeitsmatrix $Q_{\mathrm{s}}$:
\begin{align}
    \underline{Q}_{\mathrm{s}} &=
    \begin{bmatrix}
        -0.0022\cdot 10^8 & -0.0028\cdot 10^8 \\\\
        0.0059\cdot 10^8 & -1.0588\cdot 10^8
    \end{bmatrix}
    \label{eq:Gleichung18}
\end{align}
\newline
Die Matrix ist quadratisch, somit ist die Determinante maßgebend und muss ungleich Null sein.\\
\newline
Determinante von $\underline{Q}_{\mathrm{s}}$:
\begin{align*}
    \underline{\underline{det(\underline{Q}_{\mathrm{s}})}} &= \underline{\underline{ 2.2873\cdot 10^{13}}}
\end{align*}
\newline
Das System ist steuerbar, da gilt: $det(\underline{Q}_{\mathrm{s}}) \neq 0$.
