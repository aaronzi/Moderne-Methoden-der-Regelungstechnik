\section{Modellierung des Buck-Converters} \label{sec:Modellierung}

\subsection{Ströme} \label{sec:Stroeme}

Zur Modellierung des Ausgangsstroms einer PV-Anlage sind Kenntnisse über die einzelnen Anlagenströme notwendig. Zur ersten Betrachtung wird der \textbf{Strom einer einzelnen Zelle} dargelegt. Dieser berechnet sich zu

\begin{align}
    i_{\mathrm{pv,z}}(v_{\mathrm{pv,z}}, S, T_{\mathrm{c}}) &= i_{\mathrm{ph}}(S, T_{\mathrm{c}}) - i_{\mathrm{d}}(v_{\mathrm{pv,z}}, S, T_{\mathrm{c}}) - \frac{v_{\mathrm{pv,z}}}{R_{\mathrm{h}}}
    \label{eq:Gleichung2}
\end{align}
\newline
mit Strom $i_{\mathrm{ph}}$ unter Berücksichtigung der äußeren Bestrahlung dargestellt in nachfolgender Gleichung:

\begin{align*}
    i_{\mathrm{ph}}(S, T_{\mathrm{c}}) &= \cfrac{S}{S_{\mathrm{STC}}}\cdot i_{\mathrm{ph,sc,STC}}\cdot\left(1+\alpha_{\mathrm{T}}\cdot\left(T_{\mathrm{c}}-T_{\mathrm{c,STC}}\right)\right)
\end{align*}
\newline
Der \textbf{Sättigungsstrom} $i_{\mathrm{s}}$ des Dioden-Diffusionseffekts wird berechnet über

\begin{align*}
    i_{\mathrm{s}}(S, T_{\mathrm{c}}) &= \cfrac{i_{\mathrm{ph}}(S, T_{\mathrm{c}})-\cfrac{v_{\mathrm{oc}}(T_{\mathrm{c}})}{R_{\mathrm{h}}}}{e^{\cfrac{v_{\mathrm{oc}}(T_{\mathrm{c}})}{A_{\mathrm{n}}\cdot v_{\mathrm{T,STC}}}}-1}
\end{align*}
\newline
und wird eingesetzt in die Formel zum \textbf{Diodenstrom} $i_{\mathrm{d}}$:

\begin{align*}
    i_{\mathrm{d}}(v_{\mathrm{pv,z}}, S, T_{\mathrm{c}}) &= i_{\mathrm{s}}(S, T_{\mathrm{c}})\cdot \left(e^{\cfrac{v_{\mathrm{pv,z}}}{A_{\mathrm{n}}\cdot v_{\mathrm{T,STC}}}}-1\right)
\end{align*}
\newline
Weiterhin errechnet sich die \textbf{thermische Leerlaufspannung} $v_{\mathrm{oc}}$ pro Zelle über

\begin{align*}
    v_{\mathrm{oc}}(T_{\mathrm{c}}) &= v_{\mathrm{oc,STC}}\cdot\left(1+\beta_{\mathrm{T}}\cdot\left(T_{\mathrm{c}}-T_{\mathrm{c,STC}}\right)\right).
\end{align*}
\newline
Durch das Einsetzen der vorangegangenen Gleichungen in \autoref{eq:Gleichung2} folgt für den \textbf{Strom einer Zelle} zu

\begin{empheq}[box=\widefbox]{align}
    i_{\mathrm{pv,z}}(v_{\mathrm{pv,z}}, S, T_{\mathrm{c}}) &= i_{\mathrm{ph}}(S, T_{\mathrm{c}})-i_{\mathrm{s}}(S, T_{\mathrm{c}})\cdot\left(e^{\cfrac{v_{\mathrm{pv,z}}}{A_{\mathrm{n}}\cdot v_{\mathrm{T,STC}}}}-1\right)-\cfrac{v_{\mathrm{pv,z}}}{R_{\mathrm{h}}}.
    \label{eq:Gleichung3}
\end{empheq}
\newline
Um Stromgleichungen für eine PV-Anlage bestehend aus mehreren Modulen zu motivieren, werden die Anzahl der \textbf{seriellen und parallelen Zellen} pro Modul mit der Anzahl der seriellen und parallelen Module multipliziert.

\begin{align*}
    N_{\mathrm{s}} &= N_{\mathrm{cell,s}}\cdot N_{\mathrm{mod,s}} \\ \nonumber \\
    N_{\mathrm{p}} &= N_{\mathrm{cell,p}}\cdot N_{\mathrm{mod,p}}
\end{align*}
\newline
Aus der Erkenntnis der vorangegangenen zwei Gleichungen resultiert für \textbf{Strom und Spannung der PV-Anlage}:

\begin{align*}
    i_{\mathrm{pv}} &= N_{\mathrm{p}}\cdot i_{\mathrm{pv,z}} \\ \nonumber \\
    v_{\mathrm{pv}} &= N_{\mathrm{s}}\cdot v_{\mathrm{pv,z}}
\end{align*}
\newline
Durch das Einfügen der Zusammenhänge in \autoref{eq:Gleichung3} resultiert der \textbf{Gesamtstrom der PV-Anlage} zu:

\begin{empheq}[box=\widefbox]{align}
    i_{\mathrm{pv}}(v_{\mathrm{pv}}, S, T_{\mathrm{c}}) &= N_{\mathrm{p}}\cdot i_{\mathrm{ph}}(S, T_{\mathrm{c}})-N_{\mathrm{p}}\cdot i_{\mathrm{s}}(S, T_{\mathrm{c}})\cdot\left(e^{\cfrac{v_{\mathrm{pv}}}{N_{\mathrm{s}}\cdot A_{\mathrm{n}}\cdot v_{\mathrm{T,STC}}}}-1\right)-\cfrac{N_{\mathrm{p}}\cdot v_{\mathrm{pv}}}{N_{\mathrm{s}}\cdot R_{\mathrm{h}}}
    \label{eq:Gleichung4}
\end{empheq}
\newline
Die Parameter $S$ (Eingangsstrahlung), $T_{\mathrm{c}}$ (Temperatur der Zellen) und $v_{\mathrm{pv}}$ (PV-Spannung) werden im späteren Verlauf spezifiziert und zur Berechnung herangezogen.

\subsection{Induktivitäten und Kapazitäten} \label{sec:Induktivitaten_und_Kapazitaeten}

Zur Auslegung der Induktivitäten und Kapazitäten des Buck-Converters am MPP werden der Duty Cycle ($d$), die Schaltfrequenz ($f_{\mathrm{SW}}$), der Strom ($i_{\mathrm{PV,MPP}}$) und die Spannung ($v_{\mathrm{PV,MPP}}$) als bekannt vorausgesetzt. Im ersten Schritt werden die \textbf{maximal zulässigen Strom- und Spannungsschwankungen an den Bauteilen} berechnet. Dabei wird eine Schwankung von $\pm$ 0.5 $\%$ zugelassen.

\begin{align*}
    \Delta i_{\mathrm{L}} &= 0.005 \cdot i_{\mathrm{PV,MPP}} \\\\
    \Delta v_{\mathrm{PV}} &= 0.005 \cdot v_{\mathrm{PV,MPP}}
\end{align*}
\newline
Zur Berechnung der \textbf{Induktivität ($L$)} und der \textbf{Kapazität ($C$)} werden \autoref{eq:Gleichung5} und \autoref{eq:Gleichung6} angewendet.

\begin{align}
    L &= \cfrac{v_{\mathrm{DC}}\cdot(1-d)}{\Delta i_{\mathrm{L}}\cdot f_{\mathrm{sw}}} \label{eq:Gleichung5}\\ \nonumber \\
    C &= \cfrac{i_{\mathrm{PV,MPP}}\cdot(1-d)}{\Delta v_{\mathrm{PV}}\cdot f_{\mathrm{sw}}} \label{eq:Gleichung6}
\end{align}
\newline
Die für die Simulation angesetzten Bauteilgrößen resultieren zu

\begin{empheq}[box=\widefbox]{align}
     \underline{\underline{L}} &= \cfrac{900V\cdot(1-0.8579)}{0.005\cdot 2902.13A\cdot 5kHz} \approx \underline{\underline{1.76mH}} \label{eq:Gleichung7} \\ \nonumber \\
     \underline{\underline{C}} &= \cfrac{2902.13A\cdot (1-0.8579)}{0.005\cdot 1049.13V\cdot 5kHz} \approx \underline{\underline{15.7mF}}. \label{eq:Gleichung8}
\end{empheq}
