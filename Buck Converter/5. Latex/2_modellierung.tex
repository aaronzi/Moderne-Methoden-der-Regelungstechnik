\section{Modellierung des Buck-Converters}
\label{sec:Modellierung}
\subsection{Ströme}
\label{sec:Stroeme}
Zur Modellierung des Ausgangsstroms einer PV-Anlage sind Kenntnisse über die einzelnen Anlagenströme notwendig. Zur ersten Betrachtung werden die Ströme einer Zelle dargelegt.\\
\newline
Gesamtstrom einer Zelle:
\begin{align}
    i_{\mathrm{pv,z}}(v_{\mathrm{pv,z}}, S, T_{\mathrm{c}}) &= i_{\mathrm{ph}}(S, T_{\mathrm{c}}) - i_{\mathrm{d}}(v_{\mathrm{pv,z}}, S, T_{\mathrm{c}}) - \frac{v_{\mathrm{pv,z}}}{R_{\mathrm{h}}}
    \label{eq:Gleichung2}
\end{align}
\newline
mit:\\
\newline
Strom $i_{\mathrm{ph}}$ aufgrund äußerer Bestrahlung:
\begin{align*}
    i_{\mathrm{ph}}(S, T_{\mathrm{c}}) &= \frac{S}{S_{\mathrm{STC}}}\cdot i_{\mathrm{ph,sc,STC}}\cdot\left(1+\alpha_{\mathrm{T}}\cdot\left(T_{\mathrm{c}}-T_{\mathrm{c,STC}}\right)\right)
\end{align*}
\newline
Sättigungsstrom $i_{\mathrm{s}}$ des Dioden-Diffusionseffekts:
\begin{align*}
    i_{\mathrm{s}}(S, T_{\mathrm{c}}) &= \frac{i_{\mathrm{ph}}(S, T_{\mathrm{c}})-\frac{v_{\mathrm{oc}}(T_{\mathrm{c}})}{R_{\mathrm{h}}}}{e^{\frac{v_{\mathrm{oc}}(T_{\mathrm{c}})}{A_{\mathrm{n}}\cdot v_{\mathrm{T,STC}}}}-1}
\end{align*}
\newline
Diodenstrom $i_{\mathrm{d}}$:
\begin{align*}
    i_{\mathrm{d}}(v_{\mathrm{pv,z}}, S, T_{\mathrm{c}}) &= i_{\mathrm{s}}(S, T_{\mathrm{c}})\cdot \left(e^{\frac{v_{\mathrm{pv,z}}}{A_{\mathrm{n}}\cdot v_{\mathrm{T,STC}}}}-1\right)
\end{align*}
\newline
Thermische Leerlaufspannung $v_{\mathrm{oc}}$ pro Zelle:
\begin{align*}
    v_{\mathrm{oc}}(T_{\mathrm{c}}) &= v_{\mathrm{oc,STC}}\cdot\left(1+\beta_{\mathrm{T}}\cdot\left(T_{\mathrm{c}}-T_{\mathrm{c,STC}}\right)\right)
\end{align*}
\newline
Durch das Einsetzen der vorangegangenen Gleichungen in \autoref{eq:Gleichung2} folgt für den Strom einer Zelle:
\begin{align}
    i_{\mathrm{pv,z}}(v_{\mathrm{pv,z}}, S, T_{\mathrm{c}}) &= i_{\mathrm{ph}}(S, T_{\mathrm{c}})-i_{\mathrm{s}}(S, T_{\mathrm{c}})\cdot\left(e^{\frac{v_{\mathrm{pv,z}}}{A_{\mathrm{n}}\cdot v_{\mathrm{T,STC}}}}-1\right)-\frac{v_{\mathrm{pv,z}}}{R_{\mathrm{h}}}
    \label{eq:Gleichung3}
\end{align}
\newline
Um Stromgleichungen für eine PV-Anlage bestehend aus mehreren Modulen zu motivieren, werden die Anzahl der seriellen und parallelen Zellen pro Modul mit der Anzahl der seriellen und parallelen Module multipliziert. Folgender Zusammenhang gilt:\\
\newline
Berechnung der Gesamtzahl serieller ($N_{\mathrm{s}}$) und paralleler ($N_{\mathrm{p}}$) Zellen pro PV-Anlage:
\begin{align*}
    N_{\mathrm{s}} &= N_{\mathrm{cell,s}}\cdot N_{\mathrm{mod,s}} \\ \nonumber \\
    N_{\mathrm{p}} &= N_{\mathrm{cell,p}}\cdot N_{\mathrm{mod,p}}
\end{align*}
\newline
Aus der Erkenntnis der vorangegangenen zwei Gleichungen resultiert für Strom und Spannung der PV-Anlage:\\
\newline
Strom und Spannung der PV-Anlage:
\begin{align*}
    i_{\mathrm{pv}} &= N_{\mathrm{p}}\cdot i_{\mathrm{pv,z}} \\ \nonumber \\
    v_{\mathrm{pv}} &= N_{\mathrm{s}}\cdot v_{\mathrm{pv,z}}
\end{align*}
\newline
Durch das Einfügen der Zusammenhänge in \autoref{eq:Gleichung3} resultiert der Gesamtstrom der PV-Anlage zu:\\
\newline
Gesamtstrom der PV-Anlage:
\begin{align}
    i_{\mathrm{pv}}(v_{\mathrm{pv}}, S, T_{\mathrm{c}}) &= N_{\mathrm{p}}\cdot i_{\mathrm{ph}}(S, T_{\mathrm{c}})-N_{\mathrm{p}}\cdot i_{\mathrm{s}}(S, T_{\mathrm{c}})\cdot\left(e^{\frac{v_{\mathrm{pv}}}{N_{\mathrm{s}}\cdot A_{\mathrm{n}}\cdot v_{\mathrm{T,STC}}}}-1\right)-\frac{N_{\mathrm{p}}\cdot v_{\mathrm{pv}}}{N_{\mathrm{s}}\cdot R_{\mathrm{h}}}
    \label{eq:Gleichung4}
\end{align}
\newline
Die Parameter $S$ (Eingangsstrahlung), $T_{\mathrm{c}}$ (Temperatur der Zellen) und $v_{\mathrm{pv}}$ (PV-Spannung) werden im späteren Verlauf spezifiziert und zur Berechnung herangezogen.

\subsection{Induktivitäten und Kapazitäten}
\label{sec:Induktivitaten_und_Kapazitaeten}
Zur Auslegung der Induktivitäten und Kapazitäten des Buck-Converters am MPP werden der Duty Cycle ($d$), die Schaltfrequenz ($f_{\mathrm{SW}}$), der Strom ($i_{\mathrm{PV,MPP}}$) und die Spannung ($v_{\mathrm{PV,MPP}}$) als bekannt vorausgesetzt. Im ersten Schritt werden die maximal zulässigen Strom- und Spannungsschwankungen an den Bauteilen berechnet. Dabei wird eine Schwankung von $\pm$ 0.5 $\%$ zugelassen.\\
Strom- und Spannungsschwankungen an den Bauteilen:
\begin{align*}
    \Delta i_{\mathrm{L}} &= 0.005 \cdot i_{\mathrm{PV,MPP}} \\\\
    \Delta v_{\mathrm{PV}} &= 0.005 \cdot v_{\mathrm{PV,MPP}}
\end{align*}
\newline
Zur Berechnung der Induktivität (L) und der Kapazität (C) werden die \autoref{eq:Gleichung5} und \autoref{eq:Gleichung6} angewendet.\\
\newline
Berechnung von L und C:
\begin{align}
    L &= \frac{v_{\mathrm{DC}}\cdot(1-d)}{\Delta i_{\mathrm{L}}\cdot f_{\mathrm{sw}}} \label{eq:Gleichung5}\\ \nonumber \\
    C &= \frac{i_{\mathrm{PV,MPP}}\cdot(1-d)}{\Delta v_{\mathrm{PV}}\cdot f_{\mathrm{sw}}} \label{eq:Gleichung6}
\end{align}
\newline
Die für die Simulation angesetzten Bauteilgrößen resultieren zu:\\
\newline
Berechnung der Bauteilgrößen:
\begin{align}
     \underline{\underline{L}} &= \frac{900V\cdot(1-0.8579)}{0.005\cdot 2902.13A\cdot 5kHz} \approx \underline{\underline{1.76mH}} \label{eq:Gleichung7} \\ \nonumber \\
     \underline{\underline{C}} &= \frac{2902.13A\cdot (1-0.8579)}{0.005\cdot 1049.13V\cdot 5kHz} \approx \underline{\underline{15.7mF}} \label{eq:Gleichung8}
\end{align}
