\section{Linearisiertes Zustandsraummodell}

\subsection{Linearisierungsvorschrift}

Das Verhalten des nichtlinearen Systems ist für große Änderungen des Eingangssignals nicht vorhersehbar. Um dennoch Aussagen über das Systemverhalten treffen zu können, wird das nichtlineare Zustandsraummodell mithilfe der Taylorreihenentwicklung um eine Ruhelage ($\underline{x}^{*}$) linearisiert. Die nichtlinearen Restglieder $\underline{R}(\Delta{\underline{x}^2}, \Delta{\underline{u}^2})$ werden zu Null angenommen.
Durch die Linearisierung wird das Systemverhalten für kleine Änderungen um die Ruhelage kontrollierbar.\\\\
Taylorreihenentwicklung für Linearisierung:

\begin{align}\label{eq:Gleichung27}
    \begin{split}
        \dot{\underline{x}}^{*}+\Delta{\dot{\underline{x}}} &=\underline{f}(\underline{x}^{*}+\Delta{\underline{x}},\underline{u}^{*}+\Delta{\underline{u}})\\
        &=\underline{f}(\underline{x}^{*},\underline{u}^{*})+\left[\frac{\partial f_{\mathrm{i}}}{\partial x_{\mathrm{j}}}\right]_{(\underline{x}^{*}, \underline{u}^{*})}\cdot\Delta{\underline{x}}+\left[\frac{\partial f_{\mathrm{i}}}{\partial u_{\mathrm{j}}}\right]_{(\underline{x}^{*},\underline{u}^{*})}\cdot\Delta{\underline{u}}+\underline{R}(\Delta{\underline{x}^2}, \Delta{\underline{u}^2})
    \end{split}
\end{align}

Durch die Annahme über das Verhalten der nichtlinearen Restglieder folgt die Struktur des linearen Zustandraummodells aus \autoref{eq:Gleichung28}.

\begin{align}\label{eq:Gleichung28}
    \begin{split}
        \Delta\dot{\underline{x}} &= \left[\frac{\partial f_{\mathrm{i}}}{\partial x_{\mathrm{j}}}\right]_{(\underline{x}^{*}, \underline{u}^{*})}\cdot\Delta{\underline{x}}+\left[\frac{\partial f_{\mathrm{i}}}{\partial u_{\mathrm{j}}}\right]_{(\underline{x}^{*},\underline{u}^{*})}\cdot\Delta{\underline{u}}\\   
        \Delta{\underline{y}} &= \left[\frac{\partial h_{\mathrm{i}}}{\partial x_{\mathrm{j}}}\right]_{(\underline{x}^{*}, \underline{u}^{*})}\cdot\Delta{\underline{x}}+\left[\frac{\partial h_{\mathrm{i}}}{\partial u_{\mathrm{j}}}\right]_{(\underline{x}^{*},\underline{u}^{*})}\cdot\Delta{\underline{u}}
    \end{split}
\end{align}

 Allgemein gefasst, wird das lineare Zustandsraummodell folgendermaßen dargestellt:

\begin{align}\label{eq:Gleichung29}
    \begin{split}
        \dot{\underline{x}} &= \underline{A}\cdot\underline{x}+\underline{B}\cdot\underline{u}\\
        \underline{y} &= \underline{C}\cdot\underline{x}+\underline{D}\cdot\underline{u}
    \end{split}
\end{align}

\subsection{Stabile und instabile Ruhelage}
Um das linearisierte Zustandsraummodell zu erhalten, werden die einzelnen Gleichungen des nichtlinearen Zustandsraummodells aus \autoref{eq:Gleichung26} nach den Zuständen $x_{\mathrm{1}}$ bis $x_{\mathrm{4}}$, sowie dem Eingang $F_{\mathrm{a}}$ partiell abgeleitet und die entsprechende Ruhelage eingesetzt. Folgende Ruhelagen werden betrachtet:\\
Hängendes Pendel:

\begin{align}\label{eq:Gleichung30}
    \begin{split}
        \underline{x}_{\mathrm{1}}^{*}=
        \begin{bmatrix}
            x_{\mathrm{1}}^{*}\\
            x_{\mathrm{2}}^{*}\\
            x_{\mathrm{3}}^{*}\\
            x_{\mathrm{4}}^{*}
        \end{bmatrix}=
        \begin{bmatrix}
            Pi\\
            0\\
            0\\
            0
        \end{bmatrix}
    \end{split}
\end{align}

Stehendes Pendel:

\begin{align}\label{eq:Gleichung31}
    \begin{split}
        \underline{x}_{\mathrm{2}}^{*}=
        \begin{bmatrix}
            x_{\mathrm{1}}^{*}\\
            x_{\mathrm{2}}^{*}\\
            x_{\mathrm{3}}^{*}\\
            x_{\mathrm{4}}^{*}
        \end{bmatrix}=
        \begin{bmatrix}
            0\\
            0\\
            0\\
            0
        \end{bmatrix}
    \end{split}
\end{align}

\subsection{Zustandsraumdarstellung (linear)}

Die linearisierten Zustandsraummodelle unter Berücksichtigung der Ruhelagen resultieren zu:\\\\
Hängendes Pendel:

\begin{align}\label{eq:Gleichung32}
    \begin{split}
        \Delta{\dot{\underline{x}}}&=
        \begin{bmatrix}
            0 & 1 & 0 & 0 \\
            -26.6505 & -0.0248 & 0 & -5.8333 \\
            0 & 0 & 0 & 1 \\
            -0.8502 & -7.916\cdot10^-4 & 0 & -2.333
        \end{bmatrix}\cdot
        \begin{bmatrix}
            \Delta{x_{\mathrm{1}}} \\ \Delta{x_{\mathrm{2}}} \\ \Delta{x_{\mathrm{3}}} \\ \Delta{x_{\mathrm{4}}}
        \end{bmatrix}+
        \begin{bmatrix}
            0 \\
            0.8333 \\
            0 \\
            0.3333
        \end{bmatrix}\cdot F_{\mathrm{a}}
        \\
        \Delta{\underline{y}} &=
        \begin{bmatrix}
            1 & 0 & 0 & 0 \\
            0 & 0 & 1 & 0 \\
            0 & 0 & 0 & 1
        \end{bmatrix}\cdot
        \begin{bmatrix}
            \Delta{x_{\mathrm{1}}}\\
            \Delta{x_{\mathrm{2}}}\\
            \Delta{x_{\mathrm{3}}}\\
            \Delta{x_{\mathrm{4}}}
        \end{bmatrix}+\underline{0}\cdot F_{\mathrm{a}}
    \end{split}
\end{align}

Stehendes Pendel:

\begin{align}\label{eq:Gleichung33}
    \begin{split}
        \Delta{\dot{\underline{x}}}&=
        \begin{bmatrix}
            0 & 1 & 0 & 0 \\
            26.6505 & -0.0248 & 0 & 5.8333 \\
            0 & 0 & 0 & 1 \\
            -0.8502 & 7.916\cdot10^-4 & 0 & -2.333
        \end{bmatrix}\cdot
        \begin{bmatrix}
            \Delta{x_{\mathrm{1}}} \\ \Delta{x_{\mathrm{2}}} \\         \Delta{x_{\mathrm{3}}} \\ \Delta{x_{\mathrm{4}}}
        \end{bmatrix}+
        \begin{bmatrix}
            0 \\
            -0.8333 \\
            0 \\
            0.3333
        \end{bmatrix}\cdot F_{\mathrm{a}}
        \\
        \Delta{\underline{y}} &=
        \begin{bmatrix}
            1 & 0 & 0 & 0 \\
            0 & 0 & 1 & 0 \\
            0 & 0 & 0 & 1
        \end{bmatrix}\cdot
        \begin{bmatrix}
            \Delta{x_{\mathrm{1}}}\\
            \Delta{x_{\mathrm{2}}}\\
            \Delta{x_{\mathrm{3}}}\\
            \Delta{x_{\mathrm{4}}}
        \end{bmatrix}+\underline{0}\cdot F_{\mathrm{a}}
    \end{split}
\end{align}

\clearpage

\subsection{Überprüfung der Steuerbarkeit} \label{sec:Steuerbarkeit}

Die Steuerbarkeit eines Systems ist gegeben, wenn unter Berücksichtigung der Eingangsgröße $\underline{u}(t)$ das System von einem Anfangszustand $\underline{x}_{\mathrm{0}}$ in einen beliebigen Endzustand $\underline{x}_{\mathrm{e}}$ gebracht werden kann. Dies wird mathematisch für quadratische Matrizen mithilfe der Determinante und für nicht-quadratische Matrizen mit dem Rang der Steuerbarkeitsmatrix $Q_{\mathrm{s}}$ nachgewiesen. Die Berechnung der Steuerbarkeitsmatrix erfolgt durch \autoref{eq:Gleichung34}.

\begin{align}\label{eq:Gleichung34}
    \underline{Q}_{\mathrm{s}} &= \left(\underline{B} \quad \underline{A}\cdot\underline{B} \quad ... \quad \underline{A}^{(n-1)}\cdot\underline{B}\right)
\end{align}

Zur Überprüfung werden die System- und Eingangsmatrix des linearisierten Modells um die instabile Ruhelage (stehendes Pendel) eingesetzt. Die Steuerbarkeitsmatrix $Q_{\mathrm{s}}$ ist quadratisch, d.h. die Determinante muss ungleich Null sein.\\\\
Steuerbarkeitsmatrix des Systems:

\begin{align}\label{eq:Gleichung35}
    \underline{Q}_{\mathrm{s}} &=
    \begin{bmatrix}
        0 & -0.8333 & 1.9651 & -26.7984 \\
        -0.8333 & 1.9651 & -26.7984 & 67.7740 \\
        0 & 0.3333 & -0.7784 & 2.5264 \\
        0.3333 & -0.7784 & 2.5264 & -7.5869
    \end{bmatrix}
\end{align}

Die Determinante folgt zu: 46,41, d.h. das System ist steuerbar.