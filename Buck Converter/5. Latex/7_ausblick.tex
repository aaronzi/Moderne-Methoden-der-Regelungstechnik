\section{Ausblick} \label{sec:ausblick}

Wie bereits in \autoref{sec:Reglervalidierung} aufgefallen ist, schwingt das System sehr stark bei Reglern mit einfacher Zustandsrückführung und Zustandsreglern mit Referenzwertvorsteuerung. Grund da für ist der Betrag des Imaginärteils der Polstellen, welcher um ein Vielfaches größer ist als der Betrag des Realteils. Um den Imaginärteil zu verkleiner besteht die Möglichkeit die Polregion über LMI's weiter einzuschränken. Der aktuelle Ansatz nutzt lediglich eine Einschränkung über die Vorgabe eines $\alpha$-Wertes, um die Polstellen links einer vorgegebenen Position auf der Real-Achse zu platzieren.\\
Das bisherige LMI könnte über zusätzliche Vorgaben erweitert werden, um die Polregion zu verkleinern \bzw zu optimieren. Dazu wird ein Parameter $r$ für die Auslegung eines Kreisradius um den Koordinatenursprung eingeführt sowie ein $\theta$, um einen Sektor links der Imaginärachse über die Wahl eines Winkels festzulegen. Die sich ergebende Polregion ist in \autoref{fig:Bild27} dargestellt.\\

\begin{figure}[H]
    \centering
    \begin{tikzpicture}[domain=0:0]
        \draw[very thin,color=black] (-0.1,-1.1);                               % Umgebung
        \draw[dashed] (0,-3) arc(270:90:3) -- cycle;                            % Halbkreis
        \draw[-stealth] (0,0) -- (-1.24,2.74);                                  % r-Pfeil
        \node[text width=1cm] at (0, 1.6) {$r$};                                % r
        \draw[dashed] (-0.8,-4) -- (-0.8,4);                                    % alpha-Grenze
        \draw[dashed] (0,0) -- (-3,3);                                          % +theta-Grenze
        \draw[dashed] (0,0) -- (-3,-3);                                         % -theta-Grenze
        \draw[-stealth] (0,-2.5) -- (-0.8,-2.5) node[midway, above] {$\alpha$}; % alpha-Pfeil
        \node[text width=1cm] at (0.6, -0.3) {0};                               % Koordinatenursprung
        \draw[-stealth] (-0.7,0) to [bend left] (-0.5,0.5);                     % theta-Pfeil
        \node[text width=1cm] at (-0.1, 0.2) {$\theta$};                        % theta
        \draw[] (-0.8,-0.8) -- (-0.8,0.8);                                      % Begrenzung rechte Seite des Bereichs
        \node[text width=1cm] at (0, 1.6) {$r$};                                % r
        \begin{scope}
            \clip (-5,-4) rectangle (-0.8,4);
            \draw[pattern={crosshatch}, pattern color=grey] (0,0) -- (-2.12,2.12) arc[start angle=135, delta angle=90,radius=3] -- (0,0); % gefüllter Bereich
        \end{scope}
        \node[text width=3cm] at (-1.2,0.5) {$S(\alpha,r,\theta)$};             % Beschriftung Bereich
        \draw[->] (-4.2,0) -- (2,0) node[right] {$Re$};                         % X-Achse
        \draw[->] (0,-4) -- (0,4) node[above] {$Im$};                           % Y-Achse
    \end{tikzpicture}
    \caption[Polregion bei erweitertem LMI]{Polregion bei Erweiterung des LMI zu $S(\alpha,r,\theta)$}
    \label{fig:Bild27}
\end{figure}

Die Formulierung der erweiterten LMI ist nachfolgend dargestellt:

\begin{align}
    \begin{split}
        \Gamma^1 &= AX + XA^T - BM -M^TB^T + 2\alpha X \\
        \Gamma^2 &=
        \begin{pmatrix}
            (AX + XA^T - BM - M^TB^T)\sin\theta & (AX + XA^T - BM + M^TB^T)\cos\theta \\
            (XA^T - AX - BM - M^TB^T)\cos\theta & (AX + XA^T - BM - M^TB^T)\sin\theta
        \end{pmatrix} \\
        \Gamma^3 &=
        \begin{pmatrix}
            -rX & AX - BM \\
            XA^T - M^TB^T & -rX
        \end{pmatrix}
    \end{split}
\end{align}

% Gainscheduling Regler